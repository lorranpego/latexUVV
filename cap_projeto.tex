\chapter{DESCRIÇÃO DO PROJETO} \label{capitulo}

\section{Arquitetura}

\section{Descrição do Problema}

\section{Levantamento de Requisitos}

\subsection{Requisitos Funcionais e Não Funcionais}

As tabelas \ref{rf1} a seguir, apresentam um dos requisitos levantados para o sistema.

\begin{table}[htbp!]
\caption{nome da tabela}
\begin{center}
  \begin{tabularx}{\textwidth}{|c|X|c|c|c|}
      \hline
%multicolumn é a mesclagem da linha
      \multicolumn{4}{|l|}{\textbf{RF1: requisito}} & \multicolumn{1}{l|}{\textbf{Oculto:}}\\
      \hline
      \multicolumn{5}{|p{14cm}|}{\textbf{Descrição:} a descrição}\\
      \hline
      \multicolumn{5}{|c|}{\textbf{Requisitos não funcionais}}\\
      \hline
      \textbf{Nome} & \multicolumn{1}{c|}{\textbf{Restrição}} & \textbf{Categoria} & \textbf{Desejável} & \textbf{Permanente}\\
      \hline
      \textbf{NF 1.1} & requisito. &           & ( ) & (x)\\
      \hline
    \end{tabularx}
  \end{center}
  \label{rf1}
\end{table}


\subsection{Regras de Negócio}

\begin{table}[htbp!]
  \caption{Regras de Negócio}
  \centering
      \begin{tabularx}{\textwidth}{|l|X|X|}
    \hline
      \textbf{Regra} & \textbf{Nome} & \textbf{Descrição}\\
      \hline
      RN1 & Regra & Descrição.\\
      \hline
      RN2 & Regra & Descrição.\\
      \hline
      RN3 & Regra & Descrição.\\
      \hline
      RN4 & Regra & Descrição.\\
      \hline
      RN5 & Regra & Descrição.\\
      \hline
      RN6 & Regra & Descrição.\\
      \hline 
      \end{tabularx}
  \label{regra}
\end{table}


\subsection{Descrição dos Atores}


\section{Casos de Uso}


\subsection{Diagrama de Casos de Uso}


\subsection{Descrição dos Casos de Uso}

Modelo de tabela de descrição de um caso de uso.

\begin{table}[htpb!]
 \caption{Caso de Uso: Consultar Cliente}
  \begin{center}
    \begin{tabularx}{1.0\textwidth}{|X|}
    \hline
      \textbf{CSU1:} Consultar Cliente\\
      Este caso de uso descreve as funcionalidades para consultar um cliente no sistema.\\
      \hline
      \textbf{Atores:} Funcionário, Gerente\\
      \hline
      \textbf{Pré-Condição:} O usuário deverá estar logado no sistema.\\
      \hline
      \textbf{Pós-condição:} Não possui.\\
      \hline
      \textbf{Fluxo Principal:}
      \begin{enumerate}
        \item O caso de uso inicia quando o ator clicar na página consultar cliente.
        \item O sistema apresenta o botão Consultar para ser selecionado.
        \item O Ator clica no botão Consultar.
        \item O sistema realiza a busca dos dados informados na base de dados, retorna uma mensagem e então o caso de uso se encerra.
        \item Este caso de uso se encerra.
      \end{enumerate}\\
    \hline
      \textbf{Fluxo Alternativo (A1)}
      \begin{enumerate}
      \renewcommand{\labelenumi}{\alph{enumi}.}
        \item Se na etapa 2 do fluxo principal o ator informar que não deseja mais consultar um cliente,o passo seguinte é clicar no botão Cancelar e voltará para o passo nº 2, aguardando resposta do ator.
      \end{enumerate}\\
      \hline
      \textbf{Fluxo de Exceção (E1)}
      \begin{enumerate}
      \renewcommand{\labelenumi}{\alph{enumi}.}
        \item Dados não encontrados: na etapa 3 do CASO\#01 do fluxo principal, o sistema busca os clientes, informa mensagem de erro se não encontrar e posiciona o cursor do mouse no campo com erro.
      \end{enumerate}\\
    \hline
    \end{tabularx}
  \end{center}
  \label{dcu1}
\end{table}
\pagebreak


\section{Especificação da Análise}
\subsection{Diagrama de Pacotes}

\subsection{Diagrama de Classe}

\subsection{Diagrama de Entidade e Relacionamento}

\subsection{Dicionário de Dados}

Modelo de tabela de dicionário de dados.

%% ======================================Classe Funcionario=======================================================
\begin{table}[htp!]
\caption{Classe Funcionário}
\centering
\begin{tabularx}{\textwidth}{|X|X|X|}
\hline
\multicolumn{3}{|c|}{\textbf{Classe Funcionario}} \\
 \hline 
 \multicolumn{3}{|c|}{Contém as informações do funcionário} \\
 \hline
 \textbf{ATRIBUTO} & \textbf{TIPO} & \textbf{DESCRIÇÃO} \\
 \hline
 Id\_Cliente & Int & Identifica o funcionário \\
 \hline
 CPF & String & CPF do funcionário \\
 \hline
 Nm\_Funcionario & String & Nome do funcionário \\
 \hline
 Dt\_Nascimento & String & Data de nascimento do funcionário \\
 \hline
 Sexo & String & Sexo do funcionário \\
 \hline
 Cargo & String & Cargo do funcionário \\
 \hline
 Email & String & Email pessoal do funcionário \\
 \hline
 Nr\_Tel\_Cel & String & Número celular do funcionário \\
 \hline
 Nr\_Tel\_Res & String & Número residencial do funcionário \\
 \hline
 Usuario & Usuario & Usuário do funcionário \\
 \hline
 Endereco & Endereco & Endereço do funcionário \\
\hline
\end{tabularx}{}
\end{table}

\subsection{Diagrama de Atividades}

\subsection{Diagrama de Sequência}

\section{Descrição da implementação}


